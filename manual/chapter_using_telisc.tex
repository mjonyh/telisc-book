\chapter{Using TeLiSc}
\label{using}



%%%%%%%%%%%%%%%%%%%%%%%%%%%%%%% 3.1 %%%%%%%%%%%%%%%%%%%%%%%%%%%%%
\section{Power On and Shutdown}
\label{sec:3.1}

\subsection{Power On}
For using you TeLiSc environment or any other environment first you need to power it. To do this just power on your machine that it get proper electrical supply. That's it!

If you are using duel boot then you need to choose what environment are you prefer to use that session. If you want to enter TeLiSc, select TeLiSc from your grab menu. Now if you don't know about grab we discussed about it in section  

\subsection{Power Off}

It's much more easier than power on. You just need to open terminal and use the following command 

\begin{svgraybox}
\$ sudo poweroff
\end{svgraybox}


%%%%%%%%%%%%%%%%%%%%%%%%% 3.2    %%%%%%%%%%%%%%%%%%%%%%%%%%%%%%%%%%
\section{Internet Connection}
\label{3.2}


%%%%%%%%%%%%%%%%%%%%%%%%% 3.3  %%%%%%%%%%%%%%%%%%%%%%%%%%%%%%%%%%
\section{Terminal}
\label{3.3}

You already know \textit{TeLiSc} stand for what? So you should recognize the word \textit{Terminal}. Terminal is the most important and powerful part of TeLiSc. You have to do all necessary thing through Terminal. Like power off, reboot or installing any required software packages. For doning anything in TeLiSc you are completely dependent on Terminal. So You must have proper knowledge about Terminal.

Terminal 





%%%%%%%%%%%%%%%%%%%%%%%%%% 3.4  %%%%%%%%%%%%%%%%%%%%%%%%%%%%%%%%%%%%%
\section{Text Editor: VIM}
\label{3.4}



%%%%%%%%%%%%%%%%%%%%%%%%%%% 3.5 %%%%%%%%%%%%%%%%%%%%%%%%%%%%%%%%%%%

\section{Browser: Firefox}
\label{3.5}





%%%%%%%%%%%%%%%%%%%%%%% 3.6 %%%%%%%%%%%%%%%%%%%%%%%%%%%%%%%%%%%%%%
\section{Document Viewer:Evince}
\label{3.6}









%%%%%%%%%%%%%%%%%%%%%%%%% 3.7 %%%%%%%%%%%%%%%%%%%%%%%%%%%%%%%%%%%
\section{File Manager: Ranger/Thunar}
\label{3.7}
